
% Cross References in LaTeX

\documentclass{article}

\begin{document}

\section{Section One}\label{key1}

Morbi efficitur id lectus eget imperdiet. Praesent leo lorem, condimentum sit amet
pellentesque quis, cursus vel magna. Ut ac mi eu purus dictum blandit ut nec ipsum.

\section{Section Two}\label{key2}

% Section reference
Donec a quam eget risus volutpat convallis at eget sapien. Mauris et nulla id eros
interdum aliquam mattis eget ex. Donec (see Section \ref{key1}) velit turpis, molestie vel tincidunt eu,
fringilla vitae velit. In facilisis vitae ex eu facilisis.

% Reference to item entry
In the classical \emph{syllogism}
\begin{enumerate}
\item All men are mortal.\label{pre1}
\item Socrates is a man.\label{pre2}
\item So Socrates is a mortal.\label{con}
\end{enumerate}
Statements (\ref{pre1}) and (\ref{pre2}) are the \emph{premises} and
statement (\ref{con}) is the conclusion.

% Equation reference
\begin{equation}\label{sumsq}
(x+y)^2=x^2+2xy+y^2
\end{equation}
Changing $y$ to $-y$ in Equation (\ref{sumsq}) gives the following

\end{document}
