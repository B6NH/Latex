
% Typesetting Mathematics

\documentclass{article}

% Use amsmath package
\usepackage{amsmath}

% Define operator names
\DeclareMathOperator{\cl}{cl}
\DeclareMathOperator*{\esup}{ess\,sup}

\begin{document}

% Math in text
The equation representing a straight line in the Cartesian plane
is of the form $ax+by+c=0$, where $a$, $b$, $c$ are constants.

% Mathematical text
The equation representing a straight line in the Cartesian plane is of
the form \(ax+by+c=0\), where \(a\), \(b\), \(c\) are constants.

% Other method
The equation representing a straight line in the Cartesian plane is
of the form \begin{math}ax+by+c=0\end{math}, where \begin{math} a
\end{math}, \begin{math} b \end{math}, \begin{math} c \end{math} are
constants.

% Centered equation
The equation representing a straight line in the Cartesian plane is
of the form
$$
ax+by+c=0
$$
where $a$, $b$, $c$ are constants.

% Superscripts
In the seventeenth century, Fermat conjectured that if $n>2$, then
there are no integers $x$, $y$, $z$ for which
$$
x^n+y^n=z^n.
$$
This was proved in 1994 by Andrew Wiles.

% Subscripts
The sequence $(x_n)$ defined by
$$
x_1=1,\quad x_2=1,\quad x_n=x_{n-1}+x_{n-2}\;\;(n>2)
$$
is called the Fibonacci sequence.

% Superscripts and subscripts
$$
x_m^n\qquad x^n_m\qquad {x_m}^n\qquad {x^n}_m
$$

% Roots
Which is greater $\sqrt[4]{5}$ or $\sqrt[5]{4}$?

% Special symbol
For real numbers $x$ and $y$, define an operation $\circ$ by
$$
x\circ y = x+y-xy
$$
This operation is associative.

% Create and use new command
\newcommand{\vect}{(x_1,x_2,\dots,x_n)}
We often write $x$ to denote the vector $\vect$.

% Command with argument
\newcommand{\vectt}[1]{(#1_1,#1_2,\dots,#1_n)}
Vector $\vectt{a}$

% Cases
\begin{equation*}
|x| =
\begin{cases}
x & \text{if $x\ge 0$}\\
-x & \text{if $x\le 0$}
\end{cases}
\end{equation*}

% Matrices
The system of equations
\begin{align*}
x+y-z & = 1\\
x-y+z & = 1\\
x+y+z & = 1
\end{align*}
can be written in matrix terms as
\begin{equation*}
\begin{pmatrix}
1 & 1 & -1\\
1 & -1 & 1\\
1 & 1 & 1
\end{pmatrix}
\begin{pmatrix}
x\\
y\\
z
\end{pmatrix}
=
\begin{pmatrix}
1\\
1\\
1
\end{pmatrix}.
\end{equation*}
Here, the matrix
$\begin{pmatrix}
1 & 1 & -1\\
1 & -1 & 1\\
1 & 1 & 1
\end{pmatrix}$
is invertible.

% Matrix with square brackets
Square brackets
$
\begin{bmatrix}
a & b\\
c & d
\end{bmatrix}
$

% Matrix with lines
Determinant
$
\begin{vmatrix}
a & b\\
c & d
\end{vmatrix}
$

% Dots
Consider a finite sequence $X_1,X_2,\dotsc$, its sum $X_1+X_2+\dotsb$
and product $X_1X_2\dotsm$.

% Small matrices
$
\left|\begin{smallmatrix}
a & h & g\\
h & b & f\\
g & f & c
\end{smallmatrix}\right|
=0
$,
the matrix
$
\left(\begin{smallmatrix}
a & h & g\\
h & b & f\\
g & f & c
\end{smallmatrix}\right)
$
is not invertible.

% Sums
For $n$-tuples of complex numbers $(x_1,x_2,\dotsc,x_n)$ and
$(y_1,y_2,\dotsc,y_n)$ of complex numbers
\begin{equation*}
\left(\sum_{k=1}^n|x_ky_k|\right)^2\le
\left(\sum_{k=1}^{n}|x_k|\right)\left(\sum_{k=1}^{n}|y_k|\right)
\end{equation*}

% Fractions
From the binomial theorem, it easily follows that if $n$ is an even
number, then
\begin{equation*}
1-\binom{n}{1}\frac{1}{2}+\binom{n}{2}\frac{1}{2^2}-\dotsb
-\binom{n}{n-1}\frac{1}{2^{n-1}}=0
\end{equation*}

% Small fractions
Since $(x_n)$ converges to $0$, there exists a positive integer $p$
such that
\begin{equation*}
|x_n|<\tfrac{1}{2}\quad\text{for all $n\ge p$}
\end{equation*}

% Arrows
Thus we get
\begin{equation*}
0\xrightarrow{} A\xrightarrow[\text{monic}]{f}
B\xrightarrow[\hspace{7pt}\text{epi}\hspace{7pt}]{g}
C\xrightarrow{}0
\end{equation*}

% Infinity
Euler not only proved that the series
$\sum_{n=1}^\infty\frac{1}{n^2}$ converges, but also that
\begin{equation*}
\sum_{n=1}^\infty\frac{1}{n^2}=\frac{\pi^2}{6}
\end{equation*}

% Integral
Thus
$\lim\limits_{x\to\infty}\int_0^x\frac{\sin x}{x}\,\mathrm{d}x
=\frac{\pi}{2}$
and so by definition,
\begin{equation*}
\int_0^\infty\frac{\sin x}{x}\,\mathrm{d}x=\frac{\pi}{2}
\end{equation*}

% Product with two lines of subscripts
\begin{equation*}
p_k(x)=\prod_{\substack{i=1\\i\ne k}}^n
\left(\frac{x-t_i}{t_k-t_i}\right)
\end{equation*}

% Sideset command
Sideset
$\sideset{_{ll}^{ul}}{_{lr}^{ur}}\bigcup$

% Use defined operators
We denote the closure of $A$ in the subspace $Y$ of $X$ by
$\cl_Y(A)$
For $f\in L^\infty(R)$, we define
\begin{equation*}
||f||_\infty=\esup_{x\in R}|f(x)|
\end{equation*}

% Math italic boldface
In this case, we define
\begin{equation*}
\boldsymbol{a}+\boldsymbol{b}=\boldsymbol{c}
\end{equation*}

\end{document}
